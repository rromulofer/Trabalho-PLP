% Prof. Dr. Ausberto S. Castro Vera
% UENF - CCT - LCMAT - Curso de Ci\^{e}ncia da Computa\c{c}\~{a}o
% Campos, RJ,  2022
% Disciplina: Paradigmas de Linguagens de Programa\c{c}\~{a}o
% Aluno: Rômulo Souza Fernandes


\chapter{ Conceitos básicos da Linguagem Python}

Neste capítulo é apresentado alguns conceitos básicos da linguagem de programação Python, como Variáveis, constantes e tipos de dados básicos aceitos pela linguagem, que são inteiro, ponto flutuante, booleano, string e lista. Alguns dos livros indicados para iniciar o estudo sobre a linguagem de programação Python são:  \cite{Lutz2007}, \cite{Perkovic2016}, que foram os mesmos autores usamos como base para escrever grande parte desse e outros capítulos.

    %%%%%%%%=================================
    \section{Variáveis e constantes}
    %%%%%%%%=================================
	De acordo com \cite{Severance2016} o Python possui um ótimo recurso que é a manipulação de variáveis, essas variáveis são nomes atribuídos a valores, que possuem como propósito armazenar valores de forma que mais tarde possam ser recuperados. 
	Para criar variáveis é necessário fazer uma declaração por atribuição e atribuir valores a essas novas variáveis. 
	\begin{lstlisting}
	>>> texto = 'Hello, world!'
	>>> pi = 3,14159265
	>>> numero = 1500
	\end{lstlisting}
	No exemplo acima podemos observar a declaração por atribuição de 3 variáveis diferentes. A primeira está atribui uma string para uma variável chamada texto, a segunda variável atribui o valor 3,14 que é conhecido como Pi e a variável possui o nome pi. Já a terceira variável numero está recebendo o valor 1500, também por atribuição. Os usuários tem uma grande liberdade na hora de escolher os nomes das variáveis, apenas não sendo possível usar palavras reservadas da linguagem como nome de uma variável.
	
	As constantes na linguagem Python, diferente de outras linguagens de programação, não podem ser criadas de forma que seu valor não seja alterado. Na documentação existem algumas orientações caso o usuário queira criar uma constante com sintaxe de variável, uma delas é que todas as letras da variável que será utilizada como constante, deveram ser maiúsculas, em casos do nome desejado possuir espaço, deverá ser utilizado underline.
	\begin{lstlisting}
	>>> PRECO = 5
	>>> PRECO_PRODUTO = 2
	\end{lstlisting}
	O exemplo acima é o padrão recomendado pela documentação do Python.

    %%%%%%%%=================================
    \section{Tipos de Dados B\'{a}sicos}
    %%%%%%%%=================================
	A linguagem de programação Python é de tipagem dinâmica, com isso não é preciso declarar o tipo de variável, o tipo será definido através do valor que a variável receber, isso possibilita que o tipo mude no decorrer da execução do programa. Vamos falar sobre esses tipos de dados a seguir, com base no autor \cite{Severance2016}.
	
			\subsection{Inteiro}
			O tipo inteiro ou int representa caracteres numéricos inteiros positivos e negativos. Como já informado no texto, na linguagem Python não é necessário informar o tipo da variável na sua declaração, o exemplo abaixo deixará isso mais claro.
			\begin{lstlisting}
   >>> n1 = 5
   >>> n2 = 10
   >>> soma = n1 + n2
   >>> print(soma)	
   >>> print(type(soma))		
   15
   <class 'int'>
			\end{lstlisting}
			\subsection{Ponto Flutuante}
			
			\subsection{Booleano}
			
     %%%........................
            \subsection{String}
     %%%........................
            Resumidamente, uma string é uma sequência de caracteres imutável, não permitindo alterar uma string que já existe. É classificado como um item de dado simples. Para o Python, uma string é um array de caracteres ou qualquer grupo de caracteres escritos entre doble aspas ou aspas simples, por exemplo:
    \begin{lstlisting}
    >>> #Aspas simples
    >>> menssagem1 = 'Hello, World'
    >>> print (menssagem1)  
    Hello, World
    
    >>> #Aspas duplas
    >>> mensagem2 = "O dia esta chuvoso"
    >>> print (menssagem2)
    O dia esta chuvoso
    \end{lstlisting}
			
    \begin{itemize}
      \item \textit{Concatenação de strings}\\
            A união de strings é chamado de concatenação, isso pode ser feito utilizando o operador +, que possui essa função de concatenar quando usado com operandos do tipo string. O comprimento de uma string pode ser calculado utilizado o operador \texttt{len(string)}.
     \begin{lstlisting}
    >>> # concatenando 2 strings
    >>> moto = "Titan " + "150 ESD"
    >>> print (moto)
    Titan 150 ESD
    
    >>> print (len(moto))
    13
    \end{lstlisting}

      \item \textit{Operador de indexação}\\
      Utilizando o operador de indexação é possível acessar os caracteres um por um, utilizando o operador colchetes, o número dentro do colchetes é denominado index, usado para indicar a posição do caractere da variável e atribuir esse caractere à uma variável.
      Existem duas formas de indexar os caracteres de um string em Python:\\
      \begin{description}
        \item[Index com inteiros positivos] indexando a partir da esquerda, começando com 0, sendo o 0 o index do primeiro caractere da sequência.
        \item[Index com inteiros negativos] indexando a partir da direita, começando com -1, sendo -1 o último elemento da sequência, -2 sendo o penúltimo elemento da sequência, e assim sucessivamente.
      \end{description}

     \begin{lstlisting}
     >>> #Inteiro positivo
     >>> moto = 'titan'
     >>> letra = moto[0]
     >>> print(letra)
     t
     
     >>>#inteiro negativo
     >>>moto = 'titan'
     >>>letra = moto[-1]
     >>>print(letra)
     n
        \end{lstlisting}


      \item \textit{Operador de Fatias}\\
      O operador de acesso a itens de forma individual, também pode ser usado como operador de fatias, podendo assim extrair uma fatia inteira (subsequência) de caracteres de uma string. O operador de Fatias dispõe de três sintaxes:\\
      \texttt{sequencia[ inicio ]}\\
      \texttt{sequencia[ inicio : fim ]} \\
      \texttt{sequencia[ inicio : fim : step ]}\\
      onde \texttt{início, fim} e \texttt{step} são números inteiros.
     \begin{lstlisting}
    >>> sequencia = 'Linguagem Python'
    >>> print(sequencia[0:9])
    Linguagem
    
    >>> print(sequencia[0:9:8])
    Lm
        \end{lstlisting}

    \end{itemize}

			\subsection{Lista}
			Lista é uma sequência de objetos, diferente de uma string onde os valores são caracteres, na lista esses valores podem ser de qualquer tipo, até mesmo outras listas. Outra característica da lista que difere de uma string é que as listas são mutáveis, assim permitindo que o seu conteúdo seja modificado em qualquer momento, esse conteúdo das listas é chamado de elementos ou itens. Para criar uma lista é necessário colocar seus elementos apenas entre colchetes ou entre aspas dentro dos colchetes caso os elementos sejam caracteres.
			
			\begin{lstlisting}
	['verde ', 'amarelo', 'azul']
	[1, 2, 3, 4, 5]
			\end{lstlisting}
		A primeira lista do exemplo é formada por três strings, enquanto a segunda é formada por cinco números inteiros. Como já citado, é possível criar uma lista com elementos de tipos diferentes, como no exemplo a seguir, onde a lista contém uma lista, uma string, um float e um inteiro..
			
			\begin{lstlisting}
	[[1, 2], 'amarelo', 3.14, 100]
			\end{lstlisting}
		
		Também é possível atribuir valores de uma lista a variáveis.
		\begin{lstlisting}
	>>> numeros = [1, 2, 3, 4, 5]
	>>> cores = ['verde ', 'amarelo', 'azul']
	>>> lista_vazia = []
	
	>>> print(numeros, cores, lista_vazia)
	[1, 2, 3, 4, 5] ['verde ', 'amarelo', 'azul'] []
		\end{lstlisting}
	
	O acesso de elementos de uma lista é da mesma forma que o acesso de caracteres de uma string, o exemplo a seguir demonstra o funcionamento, o valor dentro dos [] é o índice.
	\begin{lstlisting}
		>>> print(cores[1])
		amarelo
	\end{lstlisting}
	Como as listas são mutáveis, é possível alterar seu conteúdo, atribuindo novos valores a itens ou mudando a ordem desses itens. A seguir temos um exemplo se como realizar essa alteração, observe que os [] a esquerda na atribuição significa o elemento que será alterado.
	\begin{lstlisting}
	>>> numeros = [1, 2, 3, 4, 5] 
	>>> numeros[0] = 6
	>>> print(numeros)
	[6, 2, 3, 4, 5]
	\end{lstlisting}

	\begin{itemize}
	\item {Operações com Listas}
	
	Novamente como nas strings, o + é o operador utilizado para concatenar listas e o * repete a lista determinada a quantidade de vezes desejada.
	
	\begin{lstlisting}
	>>> #concatenacao
	>>> numeros = [1, 2, 3]
	>>> cores = ['verde ', 'amarelo', 'azul']
	>>> concatena = numeros + cores
	>>> print(concatena)
	[1, 2, 3, 'verde ', 'amarelo', 'azul']
	
	>>> #repeticao de lista
	>>> print([1,2,3] * 3)
	[1, 2, 3, 1, 2, 3, 1, 2, 3]
	
	>>> print(['vermelho'] * 3)
	['vermelho', 'vermelho', 'vermelho']
	\end{lstlisting}

	\end{itemize}
     %%%%%%%%=================================
    \section{Tipos de Dados de Coleção}
    %%%%%%%%=================================


     %%%........................
            \subsection{Tipos Sequenciais}
     %%%........................


     %%%........................
            \subsection{Tipos Conjunto}
     %%%........................


     %%%........................
            \subsection{Tipos Mapeamento}
     %%%........................




    %%%%%%%%=================================
    \section{Estrutura de Controle e Funções}
    %%%%%%%%=================================

     %%%........................
            \subsection{O comando IF}
     %%%........................


      %%%........................
            \subsection{Laço FOR}
     %%%........................

     %%%........................
            \subsection{Laço WHILE}
     %%%........................


    %%%%%%%%======================
    \section{Módulos e pacotes}
    %%%%%%%%======================



       %%%........................
            \subsection{Módulos}
     %%%........................



          %%%........................
            \subsection{Pacotes}
     %%%........................






    Código fonte para a linguagem Python:
    \begin{lstlisting}
    number_1 = int(input('Ingresse o primeiro numero: '))
    number_2 = int(input('Ingresse o segundo numero: '))

    # Soma
    print('{} + {} = '.format(number_1, number_2))
    print(number_1 + number_2)

    # Substra\c{c}\~{a}o
    print('{} - {} = '.format(number_1, number_2))
    print(number_1 - number_2)

    # Multiplica\c{c}\~{a}o
    print('{} * {} = '.format(number_1, number_2))
    print(number_1 * number_2)

    # Divis\~{a}o
    print('{} / {} = '.format(number_1, number_2))
    print(number_1 / number_2)
    \end{lstlisting}





