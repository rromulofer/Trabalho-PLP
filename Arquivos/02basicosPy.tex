% Prof. Dr. Ausberto S. Castro Vera
% UENF - CCT - LCMAT - Curso de Ci\^{e}ncia da Computa\c{c}\~{a}o
% Campos, RJ,  2022
% Disciplina: Paradigmas de Linguagens de Programa\c{c}\~{a}o
% Aluno: Rômulo Souza Fernandes


\chapter{ Conceitos básicos da Linguagem Python}

Neste capítulo é apresentado alguns conceitos básicos da linguagem de programação Python, como Variáveis, constantes e tipos de dados básicos aceitos pela linguagem, que são inteiro, ponto flutuante, booleano, string e lista. Alguns dos livros indicados para iniciar o estudo sobre a linguagem de programação Python são:  \cite{Lutz2007}, \cite{Perkovic2016}, que foram os mesmos autores usamos como base para escrever grande parte desse e outros capítulos.

    %%%%%%%%=================================
    \section{Variáveis e constantes}
    %%%%%%%%=================================
	O Python possui um ótimo recurso que é a manipulação de variáveis, essas variáveis são nomes atribuídos a valores, que possuem como propósito armazenar valores de forma que mais tarde possam ser recuperados. 
	Para criar variáveis é necessário fazer uma declaração por atribuição e atribuir valores a essas novas variáveis. 
	\begin{lstlisting}
	>>> messagem = 'Hello, wordld!'
	>>> pi = 3,14159265
	>>> numero = 1500
	\end{lstlisting}
	No exemplo acima podemos observar a declaração por atribuição de 3 variáveis diferentes.

    %%%%%%%%=================================
    \section{Tipos de Dados B\'{a}sicos}
    %%%%%%%%=================================

     %%%........................
            \subsection{String}
     %%%........................
            Um string \'{e} uma sequ\^{e}ncia de caracteres considerado como um item de dado simples. Para Python, um string \'{e} um array de caracteres ou qualquer grupo de caracteres escritos entre doble aspas ou aspas simples. Por exemplo,
    \begin{lstlisting}
    >>> #usando aspas simples
    >>> pyStr1 = 'Brasil'
    >>> print (pyStr1)  Brasil
    >>> #usando aspas duplas
    >>> pyStr2 = "Oi, tudo bem?"
    >>> print (pyStr2)
    Oi, tudo bem?
    \end{lstlisting}

    \begin{itemize}
      \item \textit{Concatena\c{c}\~{a}o de strings}\\
            Strings podem ser concatenadas utilizando o operador +, e o seu comprimento pode ser calculado utilizado o operador \texttt{len(string)}
     \begin{lstlisting}
    >>> # concatenando 2 strings
    >>> pyStr = "Brasil" + " verde amarelo"
    >>> print (pyStr)
    Brasil verde amarelo
    >>> print (len(pyStr))
    20
    \end{lstlisting}

      \item \textit{Operador de indexa\c{c}\~{a}o}\\
      Qualquer caracter de um string o sequ\^{e}ncia de caracteres pode ser obtido utilizando o operador de indexa\c{c}\~{a}o []. Existem duas formas de indexar em Python, os caracteres de um string:\\
      \begin{description}
        \item[Index com inteiros positivos] indexando a partir da esquerda come\c{c}ando com 0 e  onde 0 \'{e} o index do primeiro caracter da sequ\^{e}ncia
        \item[Index com inteiros negativos] indexando a partir da direita come\c{c}ando com -1, e onde -1 \'{e} o \'{u}ltimo elemento da sequ\^{e}ncia, -2 \'{e} o pen\'{u}ltimo elemento da sequ\^{e}ncia, e assim sucessivamente.
      \end{description}

     \begin{lstlisting}
    >>> # Indexando strings
    >>> pyStr = "Programando"
    >>> print (len(pyStr))
    11
    >>> print (pyStr)
    Brasil verde amarelo
        \end{lstlisting}




      \item \textit{Operador de Fatias}\\
      O operador de acesso a itens (caracteres individuais) tamb\'{e}m pode ser utilizado como operador de fatias, para extrair uma fatia inteira (subsequ\^{e}ncia) de caracteres de um string. O operador de Fatias possui tr\^{e}s sintaxes:\\
      \texttt{seq[ inicio ]}\\
      \texttt{seq[ inicio : fim ]} \\
      \texttt{seq[ in\'{\i}cio : fim : step ]}\\
      onde \texttt{in\'{\i}cio, fim} e \texttt{step} s\~{a}o n\'{u}meros inteiros.
     \begin{lstlisting}
    >>> # Indexando strings
    >>> pyStr = "Programando Python"
    >>> print (len(pyStr))
    11
    >>> print (pyStr)
    Brasil verde amarelo
        \end{lstlisting}

    \end{itemize}

			\subsection{Lista}
			Em muitas situações, organizamos os dados em uma lista: uma lista de compras, uma lista de
			cursos, uma lista de contatos no seu telefone celular, uma lista de canções no seu player de áudio
			e assim por diante. Em Python, as listas normalmente são armazenadas em um tipo de objeto
			denominado lista. Uma lista é uma sequência de objetos. Os objetos podem ser de qualquer tipo:
			números, strings e até mesmo outras listas. Por exemplo, veja como atribuiríamos a
			variável animais à lista de strings que representa diversos animais:
			
			\begin{lstlisting}
	>>> animais = ['peixe ', 'gato', 'cao']
			\end{lstlisting}
		
			A variável animais é avaliada como a lista:
			
			\begin{lstlisting}
	>>> animais
	['peixe', 'gato', 'cao']
			\end{lstlisting}
			
			Em Python, uma lista é representada como uma sequência de objetos separados por vírgulas,
			dentro de colchetes. Uma lista vazia é representada como []. As listas podem conter itens de
			diferentes tipos. Por exemplo, a lista chamada coisas em
			>>> coisas = ['um', 2, [3, 4]]
			tem três itens: o primeiro é a string 'um', o segundo é o inteiro 2 e o terceiro item é a lista [3,
			4].
			Operadores de Lista
			A maioria dos operadores de string que vimos na seção anterior pode ser usada em listas de formas
			semelhantes. Por exemplo, os itens na lista podem ser acessados individualmente usando o
			operador de indexação, assim como os caracteres individuais podem ser acessados em uma string:
			>>> animais[0]
			'peixe'
			>>> animais[2]
			'cão'
			Figura 2.3 Uma lista de objetos de string. A lista animais é uma sequência de objetos. O
			primeiro objeto, no índice 0, é a string 'peixe'. Índices positivos e negativos podem ser usados,
			assim como para as strings.
			A Figura 2.3 ilustra a lista animais junto com a indexação dos itens da lista. Índices negativos
			também podem ser usados:
			>>> animais[-1]
			'cão'
			O comprimento de uma lista (ou seja, o número de itens nela) é calculado usando a
			função len():
			>>> len(animais)
			3
			Assim como as strings, as listas podem ser “adicionadas”, significando que podem
			ser concatenadas. Elas também podem ser “multiplicadas” por um inteiro k, que significa
			que k cópias da lista são concatenadas:
			>>> animais + animais
			['peixe ', 'gato', 'cão', 'peixe ', 'gato', 'cão']
			>>> animais * 2
			['peixe ', 'gato', 'cão', 'peixe ', 'gato', 'cão']
			Se você quiser verificar se a string 'coelho' está na lista, pode usar o operador in em uma
			expressão Booleana que é avaliada como True se a string 'coelho' aparecer na
			lista animais:
			>>> 'coelho' in animais
			False
			>>> 'cão' in animais
			True
			Na Tabela 2.2, resumimos o uso de alguns dos operadores de string. Incluímos na tabela as
			funções min(), max() e sum(), que podem apanhar uma lista como entrada e retornar,
			respectivamente, o menor item, o maior item, a soma dos itens da lista:
			>>> lst = [23.99, 19.99, 34.50, 120.99]
			>>> min(lst)
			19.99
			>>> max(lst)
			120.99
			>>> sum(lst)
			199.46999999999997
			Tabela 2.2 Operadores de lista e funções. Somente alguns dos operadores de lista comumente
			usados aparecem aqui. Para obter a lista completa no seu shell interativo, use a função de
			documentação help(): >>>help(list)
			Uso Explicação
			x in lst Verdadeiro se o objeto x estiver na lista lst; caso contrário, falso
			x not in lst Falso se o objeto x estiver na lista lst; caso contrário, verdadeiro
			lstA + lstB Concatenação das listas lstA e lstB
			lst * n,n * lst Concatenação de n cópias da lista lst
			lst[i] Item no índice i da lista lst
			len(lst) Comprimento da lista lst
			min(lst) Menor item na lista lst
			max(lst) Maior item na lista lst
			sum(lst) Soma dos itens na lista lst
			
			\subsection{Inteiro}
			Até aqui, vimos como usar diversos tipos de valores que o Python
			admite: int, float, bool, str e list. Nossa apresentação foi informal para enfatizar a
			técnica normalmente intuitiva que o Python utiliza para manipular valores. Porém, a intuição nos
			leva somente até aí. Nesse ponto, damos um passo atrás por um instante para entender mais
			formalmente o que significa um tipo e os operadores e métodos admitidos pelo tipo.
			Em Python, cada valor, seja um valor inteiro simples (como 3) ou um valor mais complexo
			(como a string 'Hello, World!' ou a lista ['hello', 4, 5]) é armazenado na memória
			como um objeto. É útil pensarmos em um objeto como um contêiner para o valor que fica dentro
			da memória do seu computador.
			A ideia de contêiner captura a motivação por trás dos objetos. A representação real e o
			processamento de, digamos, valores inteiros em um sistema de computação é bastante complicada.
			No entanto, a aritmética com valores inteiros é bem natural. Os objetos são contêineres para
			valores, inteiros ou não, que ocultam a complexidade do armazenamento e processamento de
			inteiros e oferece ao programador a única informação de que ele precisa: o valor do objeto e qual
			tipo de operações que podem ser aplicadas a ele.
			Tipo de Objeto
			Cada objeto tem, associado a ele, um tipo e um valor. Isso pode ser visto na Figura 2.4, que ilustra
			quatro objetos: um objeto inteiro com valor 3, um objeto de ponto flutuante com valor 3.0, um
			objeto de string com valor 'Hello World' e um objeto de lista com valor [1, 1, 2, 3,
			5, 8].
			Figura 2.4 Quatro objetos. Ilustração de quatro objetos com tipos diferentes. Cada objeto tem,
			associado a ele, um tipo e um valor.
			O tipo de um objeto indica que tipo de valores o objeto pode manter e que tipo de operações
			podem ser realizadas sobre esse objeto. Os tipos que vimos até aqui incluem o inteiro (int),
			ponto flutuante (float), Booleano (bool), string (str) e lista (list). A função type() do
			Python pode ser usada para determinar o tipo de um objeto:
			>>> type(3)
			<class 'int'>
			>>> type(3.0)
			<class 'float'>
			>>> type('Hello World')
			<class 'str'>
			>>> type([1, 1, 2, 3, 5, 8])
			<class 'list'>
			Quando usada sobre uma variável, a função type() retornará o tipo do objeto ao qual a variável
			se refere:
			>>> a = 3
			>>> type(a)
			<class 'int'>
			AVISO Variáveis Não Têm um Tipo
			É importante observar que uma variável não tem um tipo. Uma variável é apenas
			um nome. Somente o objeto ao qual ela se refere tem um tipo. Assim, quando
			vemos
			>>> type(a)
			<class 'int'>
			Na realidade, isso significa que o objeto ao qual a variável a atualmente se refere
			é do tipo inteiro.
			Enfatizamos atualmente porque o tipo de um objeto ao qual a se refere pode
			mudar. Por exemplo, se atribuirmos 3.0 à variável a:
			a = 3.0
			então a se referirá a um valor do tipo float:
			>>> type(a)
			<class 'float'>
			A linguagem de programação Python é considerada orientada a objeto porque os valores são
			sempre armazenados em objetos. Em linguagens de programação diferentes de Python, os valores
			de certos tipos não são armazenados em entidades abstratas, como objetos, mas explicitamente na
			memória. O termo classe é usado para se referir aos tipos cujos valores são armazenados em
			objetos. Como cada valor em Python é armazenado em um objeto, cada tipo Python é uma classe.
			Neste livro, usaremos classe e tipo significando a mesma coisa.
			Anteriormente neste capítulo, apresentamos diversos tipos numéricos do Python
			informalmente. Para ilustrar o conceito do tipo do objeto, agora vamos discutir seus
			comportamentos com mais detalhes.
			
			\subsection{Ponto Flutuante}
			A linguagem de programação básica do Python vem com funções como max() e sum() e
			classes como int, str e list. Embora estas não sejam, de forma alguma, todas as funções e
			classes embutidas da linguagem Python, o núcleo da linguagem Python é deliberadamente
			pequeno, para fins de eficiência e facilidade de uso. Além das funções e classes básicas, Python
			tem muitas outras funções e classes definidas na Biblioteca Padrão Python. A Biblioteca Padrão
			Python (Python Standard Library) consiste em milhares de funções e classes organizadas em
			componentes chamados módulos.
			Cada módulo contém um conjunto de funções e/ou classes relacionadas a determinado
			domínio de aplicação. Mais de 200 módulos embutidos formam juntos a Biblioteca Padrão Python.
			Cada módulo na Biblioteca Padrão contém funções e classes para dar suporte à programação de
			aplicações em um certo domínio. A Biblioteca Padrão inclui módulos para dar suporte, dentre
			outros, a:
			•Programação em rede
			•Programação de aplicação Web
			•Desenvolvimento de interface gráfica com o usuário (GUI)
			•Programação de bancos de dados
			•Funções matemáticas
			•Geradores de números pseudoaleatórios
			
			\subsection{Booleano}
			Em nosso primeiro estudo de caso, usaremos uma ferramenta gráfica para ilustrar (visualmente)
			os conceitos abordados neste capítulo: objetos, classes e métodos de classe, programação
			orientada a objeto e módulos. A ferramenta, Turtle graphics, permite que um usuário desenhe
			linhas e formas de um modo semelhante ao uso de uma caneta sobre o papel.
			DESVIO Turtle Graphics
			Turtle graphics tem uma longa história, desde a época em que o ramo da
			ciência da computação estava sendo desenvolvido. Ele fez parte da linguagem
			de programação Logo, desenvolvida por Daniel Bobrow, Wally Feurzig e
			Seymour Papert em 1966. A linguagem de programação Logo e seu recurso
			mais popular, turtle graphics, foi desenvolvida para fins de ensino de
			programação.
			A tartaruga (turtle) era originalmente um robô, ou seja, um dispositivo
			mecânico controlado por um operador de computador. Uma caneta era presa
			ao robô e deixava um rastro na superfície enquanto o robô se movia de acordo
			com as funções inseridas pelo operador.
			Turtle graphics está disponível a desenvolvedores Python através do módulo turtle. No
			módulo, estão definidas 7 classes e mais de 80 métodos de classe e funções. Não vamos examinar
			todos os recursos do módulo turtle. Só apresentaremos um número suficiente para nos permitir
			realizar gráficos interessantes enquanto cimentamos nosso aprendizado de objetos, classes,
			métodos de classe, funções e módulos. Fique à vontade para explorar essa divertida ferramenta
			por conta própria.
			Vamos começar importando o módulo turtle e depois instanciando um objeto Screen.
			>>> import turtle
			>>> s = turtle.Screen()
			Você notará que uma nova janela aparece com um fundo branco depois da execução do segundo
			comando. O objeto Screen é a tela de desenho na qual iremos desenhar. A classe Screen é
			definida no módulo turtle. Mais adiante, vamos apresentar alguns métodos de Screen que
			mudam a cor do fundo ou fecham a janela. No momento, só queremos começar a desenhar.
			Para iniciar nossa caneta ou, usando a terminologia turtle graphics, nossa tartaruga,
			instanciamos um objeto Turtle que chamaremos de t:
	
     %%%%%%%%=================================
    \section{Tipos de Dados de Cole\c{c}\~{a}o}
    %%%%%%%%=================================


     %%%........................
            \subsection{Tipos Sequenciais}
     %%%........................


     %%%........................
            \subsection{Tipos Conjunto}
     %%%........................


     %%%........................
            \subsection{Tipos Mapeamento}
     %%%........................




    %%%%%%%%=================================
    \section{Estrutura de Controle e Fun\c{c}\~{o}es}
    %%%%%%%%=================================

     %%%........................
            \subsection{O comando IF}
     %%%........................


      %%%........................
            \subsection{La\c{c}o FOR}
     %%%........................

     %%%........................
            \subsection{La\c{c}o WHILE}
     %%%........................


    %%%%%%%%======================
    \section{M\'{o}dulos e pacotes}
    %%%%%%%%======================



       %%%........................
            \subsection{M\'{o}dulos}
     %%%........................



          %%%........................
            \subsection{Pacotes}
     %%%........................






    C\'{o}digo fonte para a linguagem Python:
    \begin{lstlisting}
    number_1 = int(input('Ingresse o primeiro numero: '))
    number_2 = int(input('Ingresse o segundo numero: '))

    # Soma
    print('{} + {} = '.format(number_1, number_2))
    print(number_1 + number_2)

    # Substra\c{c}\~{a}o
    print('{} - {} = '.format(number_1, number_2))
    print(number_1 - number_2)

    # Multiplica\c{c}\~{a}o
    print('{} * {} = '.format(number_1, number_2))
    print(number_1 * number_2)

    # Divis\~{a}o
    print('{} / {} = '.format(number_1, number_2))
    print(number_1 / number_2)
    \end{lstlisting}





