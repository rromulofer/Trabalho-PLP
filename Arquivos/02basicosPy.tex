% Prof. Dr. Ausberto S. Castro Vera
% UENF - CCT - LCMAT - Curso de Ci\^{e}ncia da Computa\c{c}\~{a}o
% Campos, RJ,  2022
% Disciplina: Paradigmas de Linguagens de Programa\c{c}\~{a}o
% Aluno: Rômulo Souza Fernandes


\chapter{ Conceitos básicos da Linguagem Python}

Neste capítulo é apresentado alguns conceitos básicos da linguagem de programação Python, como Variáveis, constantes e tipos de dados básicos aceitos pela linguagem, que são inteiro, ponto flutuante, booleano, string e lista. Alguns dos livros indicados para iniciar o estudo sobre a linguagem de programação Python são:  \cite{Lutz2007}, \cite{Perkovic2016}, que foram os mesmos autores usamos como base para escrever grande parte desse e outros capítulos.

    %%%%%%%%=================================
    \section{Variáveis e constantes}
    %%%%%%%%=================================
	De acordo com \cite{Severance2016} o Python possui um ótimo recurso que é a manipulação de variáveis, essas variáveis são nomes atribuídos a valores, que possuem como propósito armazenar valores de forma que mais tarde possam ser recuperados. 
	Para criar variáveis é necessário fazer uma declaração por atribuição e atribuir valores a essas novas variáveis. 
	\begin{lstlisting}
	>>> texto = 'Hello, world!'
	>>> pi = 3,14159265
	>>> numero = 1500
	\end{lstlisting}
	No exemplo acima podemos observar a declaração por atribuição de 3 variáveis diferentes. A primeira está atribui uma string para uma variável chamada texto, a segunda variável atribui o valor 3,14 que é conhecido como Pi e a variável possui o nome pi. Já a terceira variável numero está recebendo o valor 1500, também por atribuição. Os usuários tem uma grande liberdade na hora de escolher os nomes das variáveis, apenas não sendo possível usar palavras reservadas da linguagem como nome de uma variável.
	
	As constantes na linguagem Python, diferente de outras linguagens de programação, não podem ser criadas de forma que seu valor não seja alterado. Na documentação existem algumas orientações caso o usuário queira criar uma constante com sintaxe de variável, uma delas é que todas as letras da variável que será utilizada como constante, deveram ser maiúsculas, em casos do nome desejado possuir espaço, deverá ser utilizado underline.
	\begin{lstlisting}
	>>> PRECO = 5
	>>> PRECO_PRODUTO = 2
	\end{lstlisting}
	O exemplo acima é o padrão recomendado pela documentação do Python.

    %%%%%%%%=================================
    \section{Tipos de Dados B\'{a}sicos}
    %%%%%%%%=================================
	A linguagem de programação Python é de tipagem dinâmica, com isso não é preciso declarar o tipo de variável, o tipo será definido através do valor que a variável receber, isso possibilita que o tipo mude no decorrer da execução do programa. Vamos falar sobre esses tipos de dados a seguir, com base no autor \cite{Severance2016}.
     %%%........................
            \subsection{String}
     %%%........................
            Resumidamente, uma string é uma sequência de caracteres, sendo classificado como um item de dado simples. Para o Python, uma string é um array de caracteres ou qualquer grupo de caracteres escritos entre doble aspas ou aspas simples, por exemplo:
    \begin{lstlisting}
    >>> #Aspas simples
    >>> menssagem1 = 'Hello, World'
    >>> print (menssagem1)  
    Hello, World
    
    >>> #Aspas duplas
    >>> mensagem2 = "O dia esta chuvoso"
    >>> print (menssagem2)
    O dia esta chuvoso
    \end{lstlisting}
			
    \begin{itemize}
      \item \textit{Concatenação de strings}\\
            A união de strings é chamado de concatenação, isso pode ser feito utilizando o operador +, que possui essa função de concatenar quando usado com operandos do tipo string. O comprimento de uma string pode ser calculado utilizado o operador \texttt{len(string)}.
     \begin{lstlisting}
    >>> # concatenando 2 strings
    >>> moto = "Titan " + "150 ESD"
    >>> print (moto)
    Titan 150 ESD
    
    >>> print (len(moto))
    13
    \end{lstlisting}

      \item \textit{Operador de indexação}\\
      Utilizando o operador de indexação é possível acessar os caracteres um por um, utilizando o operador colchetes, o número dentro do colchetes é denominado index, usado para indicar a posição do caractere da variável e atribuir esse caractere à uma variável.
      Existem duas formas de indexar os caracteres de um string em Python:\\
      \begin{description}
        \item[Index com inteiros positivos] indexando a partir da esquerda, começando com 0, sendo o 0 o index do primeiro caractere da sequência.
        \item[Index com inteiros negativos] indexando a partir da direita, começando com -1, sendo -1 o último elemento da sequência, -2 sendo o penúltimo elemento da sequência, e assim sucessivamente.
      \end{description}

     \begin{lstlisting}
     >>> #Acessar caracteres
     >>> moto = 'titan'
     >>> letra = moto[0]
     >>> print(letra)
     t
     
    >>> # Indexando strings
    >>> pyStr = "Programando"
    >>> print (len(pyStr))
    11
    >>> print (pyStr)
    Brasil verde amarelo
        \end{lstlisting}




      \item \textit{Operador de Fatias}\\
      O operador de acesso a itens (caracteres individuais) tamb\'{e}m pode ser utilizado como operador de fatias, para extrair uma fatia inteira (subsequ\^{e}ncia) de caracteres de um string. O operador de Fatias possui tr\^{e}s sintaxes:\\
      \texttt{seq[ inicio ]}\\
      \texttt{seq[ inicio : fim ]} \\
      \texttt{seq[ in\'{\i}cio : fim : step ]}\\
      onde \texttt{in\'{\i}cio, fim} e \texttt{step} s\~{a}o n\'{u}meros inteiros.
     \begin{lstlisting}
    >>> # Indexando strings
    >>> pyStr = "Programando Python"
    >>> print (len(pyStr))
    11
    >>> print (pyStr)
    Brasil verde amarelo
        \end{lstlisting}

    \end{itemize}

			\subsection{Lista}
			Em muitas situações, organizamos os dados em uma lista: uma lista de compras, uma lista de
			cursos, uma lista de contatos no seu telefone celular, uma lista de canções no seu player de áudio
			e assim por diante. Em Python, as listas normalmente são armazenadas em um tipo de objeto
			denominado lista. Uma lista é uma sequência de objetos. Os objetos podem ser de qualquer tipo:
			números, strings e até mesmo outras listas. Por exemplo, veja como atribuiríamos a
			variável animais à lista de strings que representa diversos animais:
			
			\begin{lstlisting}
	>>> animais = ['peixe ', 'gato', 'cao']
			\end{lstlisting}
		
			A variável animais é avaliada como a lista:
			
			\begin{lstlisting}
	>>> animais
	['peixe', 'gato', 'cao']
			\end{lstlisting}
			
			Em Python, uma lista é representada como uma sequência de objetos separados por vírgulas,
			dentro de colchetes. Uma lista vazia é representada como []. As listas podem conter itens de
			diferentes tipos. Por exemplo, a lista chamada coisas em
			>>> coisas = ['um', 2, [3, 4]]
			tem três itens: o primeiro é a string 'um', o segundo é o inteiro 2 e o terceiro item é a lista [3,
			4].
			Operadores de Lista
			A maioria dos operadores de string que vimos na seção anterior pode ser usada em listas de formas
			semelhantes. Por exemplo, os itens na lista podem ser acessados individualmente usando o
			operador de indexação, assim como os caracteres individuais podem ser acessados em uma string:
			>>> animais[0]
			'peixe'
			>>> animais[2]
			'cão'
			Figura 2.3 Uma lista de objetos de string. A lista animais é uma sequência de objetos. O
			primeiro objeto, no índice 0, é a string 'peixe'. Índices positivos e negativos podem ser usados,
			assim como para as strings.
			A Figura 2.3 ilustra a lista animais junto com a indexação dos itens da lista. Índices negativos
			também podem ser usados:
			>>> animais[-1]
			'cão'
			O comprimento de uma lista (ou seja, o número de itens nela) é calculado usando a
			função len():
			>>> len(animais)
			3
			Assim como as strings, as listas podem ser “adicionadas”, significando que podem
			ser concatenadas. Elas também podem ser “multiplicadas” por um inteiro k, que significa
			que k cópias da lista são concatenadas:
			>>> animais + animais
			['peixe ', 'gato', 'cão', 'peixe ', 'gato', 'cão']
			>>> animais * 2
			['peixe ', 'gato', 'cão', 'peixe ', 'gato', 'cão']
			Se você quiser verificar se a string 'coelho' está na lista, pode usar o operador in em uma
			expressão Booleana que é avaliada como True se a string 'coelho' aparecer na
			lista animais:
			>>> 'coelho' in animais
			False
			>>> 'cão' in animais
			True
			Na Tabela 2.2, resumimos o uso de alguns dos operadores de string. Incluímos na tabela as
			funções min(), max() e sum(), que podem apanhar uma lista como entrada e retornar,
			respectivamente, o menor item, o maior item, a soma dos itens da lista:
			>>> lst = [23.99, 19.99, 34.50, 120.99]
			>>> min(lst)
			19.99
			>>> max(lst)
			120.99
			>>> sum(lst)
			199.46999999999997
			Tabela 2.2 Operadores de lista e funções. Somente alguns dos operadores de lista comumente
			usados aparecem aqui. Para obter a lista completa no seu shell interativo, use a função de
			documentação help(): >>>help(list)
			Uso Explicação
			x in lst Verdadeiro se o objeto x estiver na lista lst; caso contrário, falso
			x not in lst Falso se o objeto x estiver na lista lst; caso contrário, verdadeiro
			lstA + lstB Concatenação das listas lstA e lstB
			lst * n,n * lst Concatenação de n cópias da lista lst
			lst[i] Item no índice i da lista lst
			len(lst) Comprimento da lista lst
			min(lst) Menor item na lista lst
			max(lst) Maior item na lista lst
			sum(lst) Soma dos itens na lista lst
			
			\subsection{Inteiro}
			
			\subsection{Ponto Flutuante}
			
			
			\subsection{Booleano}
			
	
     %%%%%%%%=================================
    \section{Tipos de Dados de Cole\c{c}\~{a}o}
    %%%%%%%%=================================


     %%%........................
            \subsection{Tipos Sequenciais}
     %%%........................


     %%%........................
            \subsection{Tipos Conjunto}
     %%%........................


     %%%........................
            \subsection{Tipos Mapeamento}
     %%%........................




    %%%%%%%%=================================
    \section{Estrutura de Controle e Fun\c{c}\~{o}es}
    %%%%%%%%=================================

     %%%........................
            \subsection{O comando IF}
     %%%........................


      %%%........................
            \subsection{La\c{c}o FOR}
     %%%........................

     %%%........................
            \subsection{La\c{c}o WHILE}
     %%%........................


    %%%%%%%%======================
    \section{M\'{o}dulos e pacotes}
    %%%%%%%%======================



       %%%........................
            \subsection{M\'{o}dulos}
     %%%........................



          %%%........................
            \subsection{Pacotes}
     %%%........................






    C\'{o}digo fonte para a linguagem Python:
    \begin{lstlisting}
    number_1 = int(input('Ingresse o primeiro numero: '))
    number_2 = int(input('Ingresse o segundo numero: '))

    # Soma
    print('{} + {} = '.format(number_1, number_2))
    print(number_1 + number_2)

    # Substra\c{c}\~{a}o
    print('{} - {} = '.format(number_1, number_2))
    print(number_1 - number_2)

    # Multiplica\c{c}\~{a}o
    print('{} * {} = '.format(number_1, number_2))
    print(number_1 * number_2)

    # Divis\~{a}o
    print('{} / {} = '.format(number_1, number_2))
    print(number_1 / number_2)
    \end{lstlisting}





