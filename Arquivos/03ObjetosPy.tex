% Prof. Dr. Ausberto S. Castro Vera
% UENF - CCT - LCMAT - Curso de Ci\^{e}ncia da Computa\c{c}\~{a}o
% Campos, RJ,  2022
% Disciplina: Paradigmas de Linguagens de Programa\c{c}\~{a}o
% Aluno: Rômulo Souza Fernandes


\chapter{ Programação Orientada a Objetos com Python}



   %%%%%%%%======================
    \section{Classes e Objetos}
    %%%%%%%%======================
	 No Python uma entidade é representada por uma abstração computacional, chamada objeto, que possui os métodos e atributos que a entidade pode fazer. Como sabemos a programação orientada a objetos é um dos paradigmas da linguagem Python. Na orientação a objetos, classe é a estrutura básica, simbolizando o tipo de um objeto, assim definindo o que o objeto pode realizar e suas características  \cite{Borges2014}.

   \begin{lstlisting}
    class NomeClasse:

    def metodo1:

    def Metodo2:

    \end{lstlisting}

   %%%%%%%%======================
    \section{Operadores ou Métodos}
    %%%%%%%%======================


   %%%%%%%%======================
    \section{Herança}
    %%%%%%%%======================


   %%%%%%%%======================
    \section{Estudo de Caso: }
    %%%%%%%%====================== 