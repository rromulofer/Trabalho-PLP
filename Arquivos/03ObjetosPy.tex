% Prof. Dr. Ausberto S. Castro Vera
% UENF - CCT - LCMAT - Curso de Ci\^{e}ncia da Computa\c{c}\~{a}o
% Campos, RJ,  2022
% Disciplina: Paradigmas de Linguagens de Programa\c{c}\~{a}o
% Aluno: Rômulo Souza Fernandes


\chapter{ Programação Orientada a Objetos com Python}



   %%%%%%%%======================
    \section{Classes e Objetos}
    %%%%%%%%======================
	 No Python uma entidade é representada por uma abstração computacional, chamada objeto, que possui os atributos que são as qualidades e os métodos que são as ações que a entidade pode fazer. Como sabemos a programação orientada a objetos é um dos paradigmas da linguagem Python. Na orientação a objetos, classe é a estrutura básica, simbolizando o tipo de um objeto, assim definindo o que o objeto pode realizar e suas características.
	 
	 Na linguagem de programação Python um objeto é criado com base na classe e usando a atribuição. O construtor dessas classes que é um método especial, entra em execução quando novos objetos são criados. Esse construtor é chamado \textunderscore \textit{new}\textunderscore(), depois de chamar o construtor, para inicializar uma nova instância é chamado o método \textunderscore \textit{init}\textunderscore().
	 
	 Para que o objeto permaneça na memória é preciso de no mínimo uma referência, pois o interpretador da linguagem Python tem uma ferramenta de limpeza que exclui todos os objetos que não possuem referência, assim que os objetos sem referência são excluídos o \textunderscore \textit{done}\textunderscore(), outro método especial, é executado \cite{Borges2014}. 
	 
   \begin{lstlisting}
    class NomeClasse:

    def metodo1:

    def Metodo2:

    \end{lstlisting}

   %%%%%%%%======================
    \section{Operadores ou Métodos}
    %%%%%%%%======================


   %%%%%%%%======================
    \section{Herança}
    %%%%%%%%======================


   %%%%%%%%======================
    \section{Estudo de Caso: }
    %%%%%%%%====================== 