% Prof. Dr. Ausberto S. Castro Vera
% UENF - CCT - LCMAT - Curso de Ci\^{e}ncia da Computa\c{c}\~{a}o
% Campos, RJ,  2022
% Disciplina: Paradigmas de Linguagens de Programa\c{c}\~{a}o
% Aluno: Rômulo Souza Fernandes



\chapter{ Aplica\c{c}\~{o}es da Linguagem Python}

Neste capítulo será apresentada algumas aplicações na linguagem de programação Python, com base nos autores \cite{Perkovic2016}, 


Devem ser mostradas pelo menos CINCO aplica\c{c}\~{o}es completas da linguagem, e em cada caso deve ser apresentado:
\begin{itemize}
  \item Uma breve descri\c{c}\~{a}o da aplica\c{c}\~{a}o
  \item O c\'{o}digo completo da aplica\c{c}\~{a}o,
  \item Imagens do c\'{o}digo fonte no compilador-interpretador,
  \item Imagens dos resultados ap\'{o}s a compila\c{c}\~{a}o-interpreta\c{c}\~{a}o do c\'{o}digo fonte
  \item Links e referencias bibliogr\'{a}ficas de onde foi obtido a aplica\c{c}\~{a}o
\end{itemize}




    \section{Opera\c{c}\~{o}es b\'{a}sicas}
    
    O código a seguir apresenta algumas operações básicas da matemática, como adição, subtração, multiplicação e divisão. Essas operações serão escolhidas em um menu de opções.
	\begin{lstlisting}
# Autor: Romulo Souza Fernandes
# E-mail: 00119110559@pq.uenf.br
# Data de criacao: 28/10/22
# Ciencia da Computacao - UENF
# Disciplina: PLP
		
		
continuar_usando = "SIM"
		
while continuar_usando == "SIM":
 # Criando um menu de opcoes
 print("SELECIONE A OPERAcaO DESEJADA")
 print("+ para Adicao")
 print("- para Subtracao")
 print("* para Multiplicacao")
 print("/ para Divisao")
		
 # Interacao com o usuario
 operacao = input("\nQual operacao voce deseja realizar? ")
		
 # Criando as operacoes e as apresentacoes de respostas
		
 # Adicao
 if operacao == "+":
  a1 = float(input("\nDigite o primeiro valor: "))
  a2 = float(input("Digite o segundo valor: "))
  adicao = a1 + a2
  print("\nA soma entre", a1, "e", a2, "e:", adicao, "\n")
  print("*"*33, "\n")
  continuar_usando = input("Gostaria de fazer outra operacao? 
  ").upper()
  print("*"*33, "\n")
		
 # Subtracao
 if operacao == "-":
  b1 = float(input("\nDigite o primeiro valor: "))
  b2 = float(input("Digite o segundo valor: "))
  subtracao = b1 - b2
  print("\nA subtracao entre", b1, "e", b2, "e:", subtracao,
   "\n")
  print("*"*33, "\n")
  continuar_usando = input("Gostaria de fazer outra operacao?
   ").upper()
  print("*"*33, "\n")
		
 # Multiplicacao
 if operacao == "*":
  c1 = float(input("\nDigite o primeiro valor: "))
  c2 = float(input("Digite o segundo valor: "))
  multiplicacao = c1 * c2
  print("\nA multiplicacao entre", c1,
  "e", c2, "e:", multiplicacao, "\n")
  print("*"*33, "\n")
  continuar_usando = input("Gostaria de fazer outra operacao?
   ").upper()
  print("*"*33, "\n")
		
 # Divisao
 if operacao == "/":
  d1 = float(input("\nDigite o primeiro valor: "))
  d2 = float(input("Digite o segundo valor: "))
  while d2 == 0:  # Garantindo que d2 nao seja zero!!
  print("O segundo valor nao pode ser zero!")
  d2 = float(input("\nDigite o segundo valor (diferente 
  de zero): "))
  divisao = d1 / d2
  print("\nA divisao entre", d1, "e", d2, "e:", divisao, "\n")
  print("*"*33, "\n")
  continuar_usando = input("Gostaria de fazer outra operacao?
   ").upper()
  print("*"*33, "\n")
	\end{lstlisting}

    \section{Programas gr\'{a}ficos}
	FEITO!

    \section{Programas com Objetos}
	FEITO!
		
    \section{O algoritmo Quicksort}
	FEITO!
	
    \section{Mobile}
    FEITO!

