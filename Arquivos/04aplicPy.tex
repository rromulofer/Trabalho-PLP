% Prof. Dr. Ausberto S. Castro Vera
% UENF - CCT - LCMAT - Curso de Ci\^{e}ncia da Computa\c{c}\~{a}o
% Campos, RJ,  2022
% Disciplina: Paradigmas de Linguagens de Programa\c{c}\~{a}o
% Aluno: Rômulo Souza Fernandes



\chapter{ Aplica\c{c}\~{o}es da Linguagem Python}

Neste capítulo será apresentada 5 aplicações completas na linguagem de programação Python, com base nos autores \cite{Perkovic2016}, \cite{Borges2014}, \cite{Severance2016} e \cite{Lutz2007}. Cada caso contém:
\begin{itemize}
  \item Uma breve descri\c{c}\~{a}o da aplica\c{c}\~{a}o
  \item O c\'{o}digo completo da aplica\c{c}\~{a}o,
  \item Imagens do c\'{o}digo fonte no compilador-interpretador,
  \item Imagens dos resultados ap\'{o}s a compila\c{c}\~{a}o-interpreta\c{c}\~{a}o do c\'{o}digo fonte
  \item Links e referencias bibliogr\'{a}ficas de onde foi obtido a aplica\c{c}\~{a}o
\end{itemize}




    \section{Opera\c{c}\~{o}es b\'{a}sicas}
    
    O código a seguir apresenta algumas operações básicas da matemática que podem ser feitas na linguagem Python, como adição, subtração, multiplicação e divisão. Essas operações serão escolhidas em um menu de opções. 
	\begin{lstlisting}
# Autor: Romulo Souza Fernandes
# E-mail: 00119110559@pq.uenf.br
# Data de criacao: 28/10/22
# Ciencia da Computacao - UENF
# Disciplina: PLP
		
		
continuar_usando = "SIM"
		
while continuar_usando == "SIM":
 # Criando um menu de opcoes
 print("SELECIONE A OPERAcaO DESEJADA")
 print("+ para Adicao")
 print("- para Subtracao")
 print("* para Multiplicacao")
 print("/ para Divisao")
		
 # Interacao com o usuario
 operacao = input("\nQual operacao voce deseja realizar? ")
		
 # Criando as operacoes e as apresentacoes de respostas
		
 # Adicao
 if operacao == "+":
  a1 = float(input("\nDigite o primeiro valor: "))
  a2 = float(input("Digite o segundo valor: "))
  adicao = a1 + a2
  print("\nA soma entre", a1, "e", a2, "e:", adicao, "\n")
  print("*"*33, "\n")
  continuar_usando = input("Gostaria de fazer outra operacao? 
  ").upper()
  print("*"*33, "\n")
		
 # Subtracao
 if operacao == "-":
  b1 = float(input("\nDigite o primeiro valor: "))
  b2 = float(input("Digite o segundo valor: "))
  subtracao = b1 - b2
  print("\nA subtracao entre", b1, "e", b2, "e:", subtracao,
   "\n")
  print("*"*33, "\n")
  continuar_usando = input("Gostaria de fazer outra operacao?
   ").upper()
  print("*"*33, "\n")
		
 # Multiplicacao
 if operacao == "*":
  c1 = float(input("\nDigite o primeiro valor: "))
  c2 = float(input("Digite o segundo valor: "))
  multiplicacao = c1 * c2
  print("\nA multiplicacao entre", c1,
  "e", c2, "e:", multiplicacao, "\n")
  print("*"*33, "\n")
  continuar_usando = input("Gostaria de fazer outra operacao?
   ").upper()
  print("*"*33, "\n")
		
 # Divisao
 if operacao == "/":
  d1 = float(input("\nDigite o primeiro valor: "))
  d2 = float(input("Digite o segundo valor: "))
  while d2 == 0:  # Garantindo que d2 nao seja zero!!
  print("O segundo valor nao pode ser zero!")
  d2 = float(input("\nDigite o segundo valor (diferente 
  de zero): "))
  divisao = d1 / d2
  print("\nA divisao entre", d1, "e", d2, "e:", divisao, "\n")
  print("*"*33, "\n")
  continuar_usando = input("Gostaria de fazer outra operacao?
   ").upper()
  print("*"*33, "\n")
	\end{lstlisting}
	
	Como vemos no código, inicialmente uma variável chamada "continuar\textunderscore usando", é criada e definida como "SIM". Na estrutura de repetição While, enquanto a variável "continuar\textunderscore usando", for igual a "SIM", o laço continuará. Um menu de interação é criado dentro desse While, o menu oferece as seguintes opções de escolha, soma, subtração, multiplicação e divisão. A variável chamada "operação" tem a função de receber e guardar a opção desejada pelo usuário. Caso a opção escolhida seja "+", a operação de soma será realizada, apresentará o resultado e também irá perguntar se o usuário deseja continuar usando, caso a respostar não seja "SIM", o programa irá finalizar, funcionando da mesma forma para as outras operações, caso forem escolhidas.

    \section{Programas gr\'{a}ficos}
     O código a seguir apresenta
	\begin{lstlisting}
# Autor: Romulo Souza Fernandes
# E-mail: 00119110559@pq.uenf.br
# Data de criacao: 28/10/22
# Ciencia da Computacao - UENF
# Disciplina: PLP


# Importando todo conteudo do Tkinter
from tkinter import *

# Classe que exibe os controles na tela


class Application:
	def __init__(self, master=None):
		# Criacao do primeiro container, chamado widget1
		self.widget1 = Frame(master)
		# Informando o gerenciador de geometria pack
		self.widget1.pack()
		# Utilizando o widget label para imprimir na tela
		self.msg = Label(self.widget1, text="Primeiro
		 widget")
		self.msg["font"] = ("Verdana", "10", "italic",
		 "bold")
		self.msg.pack()
		self.sair = Button(self.widget1)
		self.sair["text"] = "Sair"
		self.sair["font"] = ("Calibri", "10")
		self.sair["width"] = 5
		self.sair["command"] = self.widget1.quit
		self.sair.pack()


# Instanciando a classe TK()
# Ela permite que os widgets sejam utilizados na aplicacao
root = Tk()

# Passando a variavel root como parametro do metodo
# construtor da classe Application
Application(root)

# Chamada do metodo para exibir na tela
root.mainloop()
	\end{lstlisting}

    \section{Programas com Objetos}
	FEITO!
		
    \section{O algoritmo Quicksort}
	FEITO!
	
    \section{Mobile}
    FEITO!

