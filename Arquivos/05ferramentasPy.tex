% Prof. Dr. Ausberto S. Castro Vera
% UENF - CCT - LCMAT - Curso de Ci\^{e}ncia da Computa\c{c}\~{a}o
% Campos, RJ,  2022
% Disciplina: Paradigmas de Linguagens de Programa\c{c}\~{a}o
% Aluno: Rômulo Souza Fernandes


\chapter{Ferramentas existentes e utilizadas}

Neste cap\'{\i}tulo devem ser apresentadas pelo menos DUAS (e no m\'{a}ximo 5) ferramentas consultadas e utilizadas para realizar o trabalho, e usar nas aplica\c{c}\~{o}es. Considere em cada caso:
\begin{itemize}
  \item Nome da ferramenta (compilador-interpretador)
  \item Endere\c{c}o na Internet
  \item Vers\~{a}o atual e utilizada
  \item Descri\c{c}\~{a}o simples (m\'{a}x 2 par\'{a}grafos)
  \item Telas capturadas da ferramenta
  \item Outras informa\c{c}\~{o}es
\end{itemize}

    \section{Editor Jupyter Notebook}


    \section{Compilador XYZ}


    \section{Interpretador Shell}
	%https://docs.python.org/pt-br/3/tutorial/interpreter.html
	%https://education.ti.com/html/webhelp/EG_TINspire/PT/Subsystems/EG_Python/Content/m_workspaces/ws_shell.HTML

    \section{Ambientes de Programação IDE PyCharm e Visual Studio Code}
     