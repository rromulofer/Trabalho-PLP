% Prof. Dr. Ausberto S. Castro Vera
% UENF - CCT - LCMAT - Curso de Ci\^{e}ncia da Computa\c{c}\~{a}o
% Campos, RJ,  2022
% Disciplina: Paradigmas de Linguagens de Programa\c{c}\~{a}o
% Aluno: Rômulo Souza Fernandes


\chapter{Conclusões}

O desenvolvimento desse trabalho sobre Python, mesmo que de forma resumida, foi muito proveitoso em diversos aspectos, após a finalização do trabalho, é notável o ganho de conhecimento sobre a linguagem de programação Python, como sua história, diversas bibliotecas nativas, ferramentas de desenvolvimento como PyCharm, Visual Studio Code e Notepad++, várias áreas de aplicação da linguagem, paradigmas que a linguagem suporta e sua sintaxe. Além do ganho de conhecimento sobre a criação de arquivos usando o TeXstudio, editor de LaTeX, JabRef para fazer as referências bibliográficas e o LaTeX em si. 

No decorrer do desenvolvimento deste trabalho surgiram algumas dificuldades, como a falta de material para basear o trabalho, como por exemplo livros disponibilizados apenas para compra ou visualização teste, oferecendo apenas algumas páginas. Além de artigos em outras línguas, que gerou alguns problemas na tradução de termos técnicos, favorecendo a demora no desenvolvimento do trabalho. Apesar de todas as dificuldades, foi possível tirar proveito, gerando conhecimento sobre novas ferramentas e meios de pesquisa de artigos bibliográficos, livros, entre outros canais de informações. Além de aprender novos termos técnicos, que são muito pertinentes na área de Ciência da Computação.

Alguns pontos não considerados que poderiam ser estudados e acrescentados nesse trabalho são: comparação da linguagem Python com outras linguagens de programação com o mesmo propósito de desenvolvimento, podendo assim obter uma visão melhor sobre as facilidades e dificuldades de uso do Python, bem como comparar a quantidade de recursos bibliográficos para pesquisas de ambas linguagens de programação, também comparar o quanto a linguagem é leve ou pesada e se sua  é sintaxe complexa ou simples. 

Por fim podemos concluir que o trabalho atingiu seu objetivo de explorar as utilidades e aplicações da linguagem Python, bem como seus pontos positivos e negativos. Se tornando um guia para a linguagem, de forma onde uma pessoa sem conhecimento prévio de Python consiga entender os pontos principais da linguagem.