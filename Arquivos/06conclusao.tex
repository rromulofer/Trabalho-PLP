% Prof. Dr. Ausberto S. Castro Vera
% UENF - CCT - LCMAT - Curso de Ci\^{e}ncia da Computa\c{c}\~{a}o
% Campos, RJ,  2022
% Disciplina: Paradigmas de Linguagens de Programa\c{c}\~{a}o
% Aluno: Rômulo Souza Fernandes


\chapter{Conclusões}

O desenvolvimento desse trabalho sobre Python, mesmo que de forma resumida, foi muito proveitoso em diversos aspectos, após a finalização do desenvolvimento do mesmo, é notável o ganho de conhecimento sobre a linguagem de programação Python, como sua história, bibliotecas, ferramentas de desenvolvimento, áreas de aplicação, paradigmas da linguagem, sintaxe. Além do ganho de conhecimento sobre a criação de arquivos usando o TeXstudio, editor de LaTeX, JabRef para fazer o referências bibliográficas e o LaTeX em si.

No decorrer do desenvolvimento deste trabalho surgiram algumas dificuldades,


Alguns pontos não considerados que poderiam ser estudados e acrescentados nesse trabalho são:

