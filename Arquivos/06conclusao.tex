% Prof. Dr. Ausberto S. Castro Vera
% UENF - CCT - LCMAT - Curso de Ci\^{e}ncia da Computa\c{c}\~{a}o
% Campos, RJ,  2022
% Disciplina: Paradigmas de Linguagens de Programa\c{c}\~{a}o
% Aluno: Rômulo Souza Fernandes


\chapter{Conclusões}

O desenvolvimento desse trabalho foi muito proveitoso em diversos aspectos, após a finalização do desenvolvimento do mesmo, é notável o ganho de conhecimento sobre a linguagem de programação Python, como sua história, bibliotecas, ferramentas de desenvolvimento, áreas de aplicação, paradigmas da linguagem, sintaxe. Além do ganho de conhecimento sobre a criação de arquivos usando o LaTeX. 

Os problemas enfrentados neste trabalho ...


O trabalho que foi desenvolvido em forma resumida ...

Aspectos n\~{a}o considerados que poderiam ser estudados ou \'{u}teis para ...



   \begin{figure}[H]
    \begin{center}
        \caption{Linguagens de programa\c{c}\~{a}o modernas} \label{ling2}
        \includegraphics[width=12cm]{Python02.png} \\
        {\tiny \sf Fonte: O autor }
    \end{center}
   \end{figure} 